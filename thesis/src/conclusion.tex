\chapter{Conclusion}

In this master thesis, the experimental study of an interferometric stabilisation of a \gls{wdm} \gls{prc} has been tackled. First, the notion of \rc has been discussed. The \rc computation paradigm is based on the concept of \gls{rnn}. It reaches state of the art performances in tasks involving the processing time-dependent inputs. It was shown that the neurons inside the \rcer need to be connected in a random way, and that when being trained, a \rcer only requires its output weights to be updated unlike classical \glspl{nn}. The mathematical model governing the discrete time evolution of a \rcer is very general and imposes very few conditions. This property allows to use physical systems other than classical silicon-based computers to implement the computing logic of a \rcer. This has already been done several times using optical setups with the different neurons being multiplexed in time and encoded in either the light intensity or in the electric field.\\

In the second part of this work, a new kind of photonic \rcer based on the \gls{wdm} of the neurons has been introduced. The main advantage of this scheme is that all the neurons are updated in the same time, enabling an increase of the processing speed. This new optical \rcer is implemented using a fibre-based setup, with the \rc computing logic taking place inside a ring cavity and with the mixing of the neurons being performed by a \gls{pm}. In this scheme, the input signal is encoded in the amplitude of the electric field emitted by a laser source. At the exit of the ring cavity, the signal is demultiplexed and the output of the \rcer is computed by linearly combining the intensity of the different frequency components of the electric field encoding the neurons. Because of practical limitations to which the \gls{pm} is subject, only a limited number of neurons can coexist in the optical cavity at the same time, which is a drawback of this scheme. \\

The last chapter of this thesis has been dedicated to the experimental study of the interferometric stabilisation properties of the optical cavity called the reservoir. It has been shown that the reservoir behaves like an interferometer and that interesting properties about the phase of the electric field inside it can be deduced by studying the power it reflects. This phase, which is the one acquired by the electric field during one round trip around the cavity, is an important parameter of the reservoir and one should be able to maintain it constant during a \rcer experiment, this is why the stabilisation of this cavity is such an important issue. The experimental setup has been presented and three main parts emerged from this description: a first block handles the shaping of the input electric field, with notably a narrow band laser and a \gls{mzm}, a second one includes the reservoir cavity and the readout of the neurons, and a last one takes care of stabilising the cavity. Different features of the reservoir not directly linked to its stabilisation have been characterised. First an analytical model of the transfer function of the reservoir, which is essentially the reflected power as a function of the phase, has been derived and validated with the help of experimental curves. After that, the effective losses experienced by an electric field inside the cavity were estimated, and finally, the influence of the modulation power on the modulation depth and by extension on the transfer function was studied. The \pdh cavity stabilisation technique has been presented. This powerful tool allows to build more robust stabilisation scheme because it is able to extract more information about the phase of the electric field than the classical transfer function. By using an additional \gls{pm} and some electronic post-processing, a more sophisticated transfer function called the error function is constructed and serves as a basis for higher quality regulations. The core of this chapter was the comparison between the different stabilisation strategies. To do so, a quantitative indicator, the phase noise, has been extensively used and was computed for each strategy and set point with different methods. The key take-away of this analysis is that the best results were always obtained with a \pdh phase modulation frequency of \SI{781}{\kilo\hertz}. For future \rcer experiments, I would suggest to use this modulation frequency.\\

An actual \rcer experiment was attempted using the values provided by the analysis presented in this thesis to get some preliminary results. Unfortunately, it was not possible to get the reservoir to work since the noise was still too important. The decrease of the integral parameter of the \gls{pid} parameter is probably not to blame, because as can be seen on the tables of appendix \ref{app-exp}, the phase noises in the last rows are not dramatically larger. One possible explanation is that the stabilisation scheme cannot handle the intensity modulation, which would make sense since it has not been designed with this constraint in mind. This gives perspectives for a future work in which the characterisation performed in this thesis would be made more relevant by including the influence of the input intensity modulation on the stabilisation performances. Also, the erratic behaviour of the phase modulation amplitude should be further investigated in order to find its origin.