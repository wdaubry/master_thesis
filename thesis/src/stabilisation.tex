\chapter{Interferometric stabilisation of reservoir cavity}

\label{ch-stabilisation}

%%%%%%%%%% INTRODUCTION %%%%%%%%%%

\section{Introduction}

In this introductory section, the concept of interferometry is presented. As the name of the chapter suggests, this technique is used to stabilise the reservoir cavity. The reason why an optical cavity needs stabilisation will appear clearer later, but basically, this is due to the fact that light is a wave and that it can interfere with itself inside the cavity. The interferences can be constructive, destructive, or can behave in any intermediate way. Moreover, it will be shown that the interferometric properties are wavelength dependent. Since several wavelengths coexist inside the reservoir, this gives a first glimpse on the complexity entailing its stabilisation. To gain some insight on interferometry, and before moving on to the study of an actual ring cavity, the features of the well known \gls{fp} interferometer are recalled. After that, it is shown that the properties studied for the \gls{fp} can be translated to ring cavities with close to no modification. Finally, under the light of the basic notions of interferometry developed, the difficulties linked to the stabilisation of the reservoir cavity, which is at the heart of the scheme introduced in this thesis, are presented.

%%% FABRY-PEROT INTERFEROMETER %%%

\subsection{Fabry-Perot interferometer}

The \gls{fp} plays an important role in modern optics as it is really ubiquitous. This can be explained by the fact that, despite its great simplicity, it can reach good performance using high reflectivity mirrors, which can be produced using nowadays technologies. In practice, a \gls{fp} cavity is simply made of two facing mirrors as can be viewed on figure \ref{fp}. On this figure, one can see the two mirrors, represented by the vertical black lines, and the different electric fields. The resonance condition, namely the regime where the transmitted electric field $E_{\text{t}}$ is maximum, can be seen intuitively as a situation where the intra-cavity field $E_1$ is in phase with the incident field $E_{\text{in}}$, which leads to the build up of a very intense intra-cavity electric field. On the other hand, the anti-resonance condition is met when $E_{\text{in}}$ and $E_1$ are out of phase. The transmissivity of the \gls{fp} interferometer, which is defined as the ratio $|E_{\text{t}}|^2/|E_{\text{in}}|^2$, is given by \cite{Perot1899}:

\begin{equation}
	\mathcal{T}(\omega) = \frac{1}{1+\mathcal{F}\sin^2{\left(\frac{\omega}{\text{FSR}}\right)}}
	\label{transmissivity}
\end{equation}

In this expression, $\mathcal{F}$ is the finesse of the cavity, $\omega$ is the angular frequency of the incident electric field, and FSR is the \acrlong{fsr} of the cavity. In a stationary regime, the energy inside the cavity does not evolve, therefore the energy carried by the incident electric field $E_{\text{in}}$ can either be transmitted or reflected, which implies that the reflectivity of the cavity which is defined as the ratio $|E_{\text{ref}}|^2/|E_{\text{in}}|^2$ is simply given by:

\begin{equation}
	\mathcal{R}(\omega) = 1 - \mathcal{T}(\omega)
	\label{reflectivity}
\end{equation}

\begin{figure}[h]
	\centering
	\includegraphics[width=.65\textwidth]{fp}
	\caption{Schematic representation of a \acrlong{fp} interferometer. $E_{\text{in}}$ is the incident electric field, $E_{\text{ref}}$ is the reflected electric field, $E_{\text{t}}$ is the transmitted electric field, $E_{1}$ is the intra-cavity electric field propagating from left to right, $E_{2}$ is the intra-cavity electric field propagating from right to left, $R$ and $T$ are the reflectivity and transmissivity of the mirrors and $L$ is the distance between them.}
	\label{fp}
\end{figure}

On figure \ref{fp-tf}, the transmissivity (right) and reflectivity (left) can be viewed. These graphs are made of peaks which are distant of the \gls{fsr} in the spectral domain. Recalling that $\text{FSR} = c/2nL$, one can see that the \gls{fsr} is linked to the length of the cavity, with $c$ the speed of light and $n$ the refractive index of the medium that could be present between the two mirrors ($nL$ is the optical path). The finesse $\mathcal{F}$ is related to the width of the peaks and depends on the reflectivity of the mirrors as $\mathcal{F} = 4R/(1-R)^2$. As the reflectivity of the mirrors tends to 1, the finesse tends to infinity, and the peaks get infinitely narrow. On the other hand, with a lower reflectivity, more energy can leak out of the cavity even outside the resonance condition. Seeing the broadening of the peaks as energy leakage will be useful when drawing a parallel between \gls{fp} and ring cavity interferometers.

\begin{figure}[h]
	\centering
	\begin{subfigure}{.5\textwidth}
		\centering
		\includegraphics[width=\textwidth]{fp-trans-tf}
	\end{subfigure}%
	\begin{subfigure}{.5\textwidth}
		\centering
		\includegraphics[width=\textwidth]{fp-ref-tf}
	\end{subfigure}
	\caption{Transmissivity $\mathcal{T}$ (left) and reflectivity $\mathcal{R}$ (right) of the cavity. Finesse $\mathcal{F}=50$. $\delta \nu$ denotes the deviation from a resonant frequency.}
	\label{fp-tf}
\end{figure}

Equation \eqref{transmissivity} and \eqref{reflectivity} indicate that $\mathcal{T}$ and $\mathcal{R}$ depend on $\omega/\text{FSR}$. This value can be rearranged as:

\begin{equation}
	\frac{\omega}{\text{FSR}} = kL = \phi
\end{equation}

Where $k$ is the wave number defined as $n \omega/c$ and $\phi$ is the phase acquired by the electric field when propagating along the cavity. Therefore, by measuring the transmitted or reflected power, one can gain information about the phase (modulo $\pi$, because the periodicity of $\mathcal{T}$ and $\mathcal{R}$ in the phase domain is $\pi$) of the electric field. This is the idea underlying interferometry.

%%% RING CAVITY %%%

\subsection{Ring cavity}

\label{subsec-ring-cavity}

A ring cavity exhibits the same behaviour as a \gls{fp} interferometer. The structure of a ring cavity interferometer is displayed on figure \ref{cavity_vs_fp}. This is a fibre-based setup in which the incident electric field $E_{\text{in}}$ penetrates the cavity from the left through a coupler. $E_{\text{ref}}$ denotes the electric field exiting the cavity and $E_1$ and $E_2$ refer to the fields entering and leaving the fibre loop, respectively. The nomenclature for the fields was chosen in such a way that the analogy with the \gls{fp} cavity can be understood. Indeed, one could see the coupler acting as the leftmost mirror from the figure \ref{fp}, and the fibre loop as the one on the right side because it turns $E_1$ into $E_2$ and dissipates energy through fibre losses, whereas for the mirror it was by leakage out of the cavity.

\begin{figure}[h]
	\centering
	\includegraphics[width=.7\textwidth]{cavity_vs_fp}
	\caption{Schematic view of a ring cavity}
	\label{cavity_vs_fp}
\end{figure}

Because of the similarities between ring cavities and \glspl{fp}, the former show the same transmissivity and reflectivity as the latter. Therefore, by measuring the reflected power, one can determine the phase acquired by the electric field inside the cavity.

%%% Challenge %%%

\subsection{Challenge}

The basic principle of interferometry has been introduced through the presentation of the \gls{fp} interferometer, and in the discussion that followed, it has been shown that a ring cavity, such as the reservoir cavity studied in this thesis, can be used as an interferometer. Moreover, it has also been showed that $\mathcal{R}$ depends on the frequency (or wavelength) and on the length of the cavity and that the phase acquired by the incident electric field after one round trip could be determined by studying the reflected power.\\

In the reservoir cavity, many different wavelengths coexist because they encode the different neurons. Furthermore, after each round trip, the phase acquired by each neuron should be a constant, as shown on equation \eqref{model-reservoir}. Recalling the phase matrix $\mathbf{\Phi}$ defined in equation \eqref{phase-matrix}, the phase factor multiplying the $k^{\text{th}}$ neuron is given by $\Phi_{kk} = \exp{(i\phi_k)}$.	The phase $\phi_k$ is given by:

\begin{equation}
	\phi_k = \beta(\omega+k\Omega) L
\end{equation}

With $\beta$ the fibre wave number, $\omega+k\Omega$ the angular frequency of the $k^{\text{th}}$ neuron and $L$ the length of the fibre loop. Because $k\Omega \ll \omega$, the wave number can be Taylor expanded: 

\begin{equation}
	\beta(\omega+k\Omega) = \beta(\omega) + \left. \frac{\partial\beta}{\partial\omega}\right\rvert_\omega k\Omega + \mathcal{O}\left((k\Omega)^2\right)
\end{equation}

By rewriting $\beta(\omega)$ and $\partial\beta/\partial\omega\rvert_\omega$ as $\beta_0$ and $\beta_1$ as it is often done in the literature, the acquired phase is given by:

\begin{equation}
	\phi_k \approx \beta_0 L + k \beta_1 \Omega L = \phi_0 + k \phi_1
\end{equation}

This means that if $\phi_1$ is an integer multiple of $\pi$, the phase factor multiplying all the neurons will be the same up to a sign:

\begin{equation}
	e^{i(\phi_0+k\phi_1)} = e^{i\phi_0}e^{ikm\pi} =(-1)^{km} e^{i\phi_0}, \quad m \in \mathbb{Z}
\end{equation}

A periodicity of $\pi$ and not $2\pi$ is considered here, because as claimed before, an interferometer can only inform about a phase modulo $\pi$. Looking for an acquired phase equal to $\pi$ is the approximately the same as taking a modulation frequency $\Omega$ for the \gls{pm} which is an integer multiple of the \gls{fsr}. Indeed, by considering $\beta_1 \approx n/c$, one can find:

\begin{equation}
	\beta_1 \Omega L \approx \frac{\Omega n L}{c} = m\pi \longrightarrow \Omega \approx \frac{m\pi c}{n L} = m2\pi ~\text{FSR}
\end{equation}

This is a legitimate assumption given the fact that in the region of interest the curve of $\beta(\omega)$ computed using the Sellmeier relations \cite{Bruckner,malitson1965interspecimen} is very linear.\\

It is not critical to modulate the phase at an angular frequency $\Omega$ which is an integer multiple of the \gls{fsr}, the reservoir can in fact operate in different regimes. Those considerations were presented to better understand the physics underlying the phase management of the reservoir.\\

The main practical challenge regarding the reservoir concerns its stabilisation. The reservoir has to be stabilised because during its operation, external elements disturb it, which deteriorates the performances and even make it unusable in the worst case. The perturbations acting on the reservoir can come from various sources, such as mechanical constraints or temperature changes for example, and can induce a phase fluctuation. This is where interferometric stabilisation comes into play. Indeed, by measuring the power reflected by the reservoir, one can infer the current phase and take the appropriate action to maintain it at a setpoint. This can be achieved using classical regulation strategies, such as a \gls{pid} regulator. Elements of regulation will be presented later. Furthermore, this needs to be performed for all the neurons at the same time. This explains why different elements regarding the relative phases between the different neurons were discussed in the previous paragraph. Finally, the last technological difficulty associated to this scheme is the fact that the electric field used to stabilise the cavity is modulated in amplitude to carry the data to be processed by the reservoir. Indeed, a regulator struggles to differentiate a variation in the reflected power due to a phase fluctuation or to the modulation of the incident field.

%%%%%%%%%% EXPERIMENTAL SETUP %%%%%%%%%%

\section{Experimental setup}

In this section, the experimental setup employed to physically implement the \gls{wdm} \gls{prc} is presented. First, it is detailed based with the help of the schematic representation of figure \ref{exp-setup}. Then, technical data about the devices involved in the experiment are given.

\begin{figure}[h]
	\centering
	\includegraphics[width=\textwidth]{exp-setup}
	\caption{Schematic representation of the experimental setup for the \gls{wdm} reservoir computer experiment}
	\label{exp-setup}
\end{figure}

The setup presented here is mostly an update on the one presented on \cite{AkroutAkram2016Pprc}. On figure \ref{exp-setup}, electrical wires and single mode polarisation maintaining fibres are represented using black and coloured lines, respectively.  The light source exciting the setup is a narrow band continuous laser, which sends a single wavelength $\lambda_0 =$ 1555 nm to the reservoir. The light goes through an isolator that prevents any backward reflection towards the laser source and then enters a polariser that ensures that only one linear polarisation mode is present inside the setup. This needs to be done because the optoelectronic devices involved in the experiment, such as the \gls{mzm} and the \glspl{pm}, are polarisation dependent. The electric field then enters a \gls{mzm} which is driven by the time dependent voltage $V_{\text{MZM}}(t)$. This signal is generated by an \gls{awg} based on the input time series $u(n)$ sent by the computer running the experiment. Since the transfer function of the \gls{mzm} is nonlinear, the signal $V_{\text{MZM}}$ can be precompensated in order to counteract the nonlinearity. Without precompensation, the voltage $V_{\text{MZM}}(t)$ is simply given by $\beta \tilde{u}(t)$, with $\beta$ the input gain $\in [0,1]$ and $\tilde{u}(t)$ the \textit{sample and hold} version of $u(n)$ already introduced in the previous chapter. A bias tension $V_0$ can also be applied to the \gls{mzm} to change its average transparency. The transfer function of the \gls{mzm} is given by :

\begin{equation}
	\sin{ \left(\frac{\pi}{2} \left(\frac{V_{\text{MZM}}(t)}{V_{\pi,\text{RF}}} + \frac{V_0}{V_{\pi, \text{DC}}} \right) \right)}
\end{equation}

With $V_{\text{MZM}} \in [-\beta V_{\pi,\text{RF}},\beta V_{\pi,\text{RF}}]$, and $V_{\pi,\text{RF}}$ and $V_{\pi,\text{DC}}$ the dynamic and static characteristic voltages of the \gls{mzm}, respectively. The modulated electric field is then pre-amplified using an \gls{edfa}, and undergoes a first phase modulation. This is required to be able to implement the \gls{pdh} stabilisation technique, which is an advanced cavity stabilisation scheme that is described with greater length in section \ref{sec-pdh}. The \gls{pm} is driven by the alternative voltage $V_{\text{PDH}} = A_{\text{PDH}} \sin{(\Omega_{\text{PDH}}t)}$ supplied by the \textit{Toptica Digilock 110} feedback controller denoted "Digilock" on the figure. This device handles every aspects related to the stabilisation of the reservoir, which is why it needs to be connected to the photodiode "PD" and to the laser, which are both involved in the regulation of the cavity (see later). The electric field then enters the reservoir through the coupler "C3". At this point, the colour used to represent the optical fibres changes from red to blue to indicate that inside the reservoir, several wavelengths are present whereas before the coupler there was only one. Inside the reservoir, the \gls{pm} is used to mix the different frequencies and is driven by an external alternative voltage $V_{\text{mod}}(t) = A_{\text{mod}} \sin{ (\Omega_{\text{mod}} t)}$. It introduces insertion losses that are compensated using the \gls{edfa}, whose pump current is adjusted to tune the feedback gain $\alpha$ that appears in equation \eqref{model-reservoir}. The electric field exiting the reservoir at the level of the coupler "C3" towards the photodiode "PD" is the reflected field, according to what was discussed in section \ref{subsec-ring-cavity}. The electric field finally illuminates the photodiode "PD", which produces a voltage proportional to its intensity. This measured signal enters the Digilock where it is compared to a reference. Based on the deviation between those two values, the Digilock outputs a control voltage that is applied to a piezoelectric crystal inside the laser cavity and that modifies its emission wavelength. The output of the reservoir exits the cavity thought the coupler "C4" and is amplified by a \gls{soa} called "BOA" for Boost Optical Amplifier to improve the \gls{snr}. A demultiplexing scheme allowing to obtain the value of each individual neuron at the same time has not been implemented yet. To overcome this limitation, the adjustable band-pass filter denoted "Wave Shaper" is used to record the evolution of only one neuron at a time. This implies that if one wants to compute the output of the reservoir $y(n)$, one has to run the same experiment once for each of the neuron, to program the Wave Shaper to filter out all the other sidebands and to save its evolution on the computer. Once this has been done, $y(n)$ can be reconstructed. As far as the learning procedure is concerned, it follows the same procedure as the one previously described, but modified to take this sequential measurements of the neurons into account. In terms of the devices used to perform this task, the photodiode "Readout PD" measures the intensity of each of the filtered neuron, and the "Digitizer" handles the conversion between continuous time dependent signals into time series usable by the computer. To conclude this description of the experimental setup, one should note that even though the couplers "C1" and "C2" seem useless, they are used in practice to monitor the electric fields when modifications are made to the optical table and that the AWG, Digilock, Wave Shaper and Digitizer are all controlled by the computer. The technical specifications of the devices used in the experiment can be found in appendix \ref{app-data}.

%%%%%%%%%% CHARACTERISATION OF THE RESERVOIR %%%%%%%%%%

\section{Characterisation of the reservoir}

To be able to develop a reliable stabilisation scheme for the reservoir, some of its characteristics need to be studied as a preliminary work.  

%%% TRANSFER FUNCTION OF THE CAVITY %%%

\subsection{Transfer function of the cavity}

% MATHEMATICAL MODEL %

\subsubsection{Mathematical model}

% SIMULATIONS %

\subsubsection{Simulations}

% EXPERIMENTAL RESULTS %

\subsubsection{Experimental results}

%%% EFFECTIVE LOSSES %%%

\subsection{Effective losses}

%%% MODULATION DEPTH %%%

\subsection{Modulation depth}

%%%%%%%%%% PDH STABILISATION TECHNIQUE %%%%%%%%%%

\section{Pound-Drever-Hall stabilisation technique}

\label{sec-pdh}

% Cite LIGO
% Cite PID

%%% INTRODUCTION %%%

\subsection{Introduction}

%%% ERROR FUNCTION %%%

\subsection{Error function}

% MATHEMATICAL MODEL %

\subsubsection{Mathematical model}

% SIMULATION %

\subsubsection{Simulation}

% EXPERIMENTAL RESULTS %

\subsubsection{Experimental results}

%%%%%%%%%% CHARACTERISATION OF THE STABILISATION PERFORMANCE FOR DIFFERENT REGIMES %%%%%%%%%%

\section{Characterisation of the stabilisation performance for different regimes}

%%% APPROACH %%%

\subsection{Approach}

%%% RESULTS %%%

\subsection{Results}