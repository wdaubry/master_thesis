\chapter*{Abstract}

Reservoir computing is a recent computing paradigm based on the concept of recurrent neural network that reaches state of the art performances in time-dependent data processing. The mathematical description of this computation scheme imposes so little constraints that it can be implemented on physical systems such as optical setups. A novel fibre-based reservoir computer relying on wavelength division multiplexing of the neurons is presented. It is supposed to exhibit an increased data processing rate compared to the previous generation of photonic reservoir computers based on time division multiplexing. In the new scheme proposed, the reservoir is physically implemented by an optical cavity and the coupling between the neurons is performed by an intra-cavity phase modulator. In order for the reservoir computer to work properly, the cavity has to be stabilised for all the neurons at the same time. To do so, an analytical model for the transfer function of the cavity is derived and validated by comparison with experimental curves and the Pound-Drever-Hall technique, which is an advanced cavity stabilisation scheme, is implemented. The cavity is characterised experimentally by finding the Pound-Drever-Hall setting that minimise the phase noise.\\

\textit{Keywords:} Reservoir computing, neural network, machine learning, photonic reservoir computing, ring cavity, cavity stabilisation, interferometric stabilisation, Pound-Drever-Hall technique